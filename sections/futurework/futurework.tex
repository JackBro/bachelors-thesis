\section{Conclusion and Future work}
\label{sec:futurework}
This chapter summarizes the thesis, draws a conclusion of the proposed solution. After that an outline is given for potential future work.
\subsection{Summary}
The thesis has contributed a kernel mode driver to tackle the problem of \gls{PMT} and gave an state-of-the art overview about current attacks and defenses. The different kinds of attacks have been grouped into two categories, external and internal modifications. As such, they were handled separately when designing the architecture of the proposed solution. \gls{WPM} and \gls{DLL} component have been introduced to handle the respective memory tampering and \gls{DLL} injection attacks. Section~\ref{sec:performance}~and~\ref{sec:security} analyzed the performance and security of the proposed solution. While the \gls{WPM} component resulted in almost no overhead, running the \gls{DLL} component resulted in a huge performance hit. During execution of the \gls{DLL} component, the process gets suspended for the time a hash is generated. It was found out, that the required time scales linearly with respect to the file size. The performance overhead is therefore the first limitation of the proposed solution. Security analysis outlined that the attacker can not bypass the driver from restricting his actions, given certain conditions. The first condition is that the privilege level of the attacker may not be higher than user privileges (i.e., no admin privileges). The second condition is, that identification of the process which should receive \gls{PMT} protection must be ensured (i.e., the attacker is not allowed to rename the process executable file name). All three limitations can possibly be solved by future research in this area.
\subsection{Future work}
This section will give an outline about possible future research involving this thesis and \gls{PMT}. Future research could develop a method to cache the generated \gls{DLL} hash. The overhead of the \gls{DLL} component will get significantly reduced, making hashing a one time cost until file changes are recognized. A method to identify a process based on a stronger heuristic than file name or file hash must be developed. This will make the proposed solution harder to bypass. In its current state, the driver will only work for \emph{Google Chrome}. Future research could develop a method to apply \gls{PMT} protection to other processes by introducing a standardized specification that allows to protect multiple independent processes. The specification outcome could then be deployed by the application vendors to make use of the installed driver. The implementation must ensure, that the introduced specification will not compromise the drivers security. Additionally, a method for safe communication between driver and application could be developed. This will allow on-the-fly updating of applications. Communication between driver and application will thus reduce the number of \glspl{DLL} that need to be hashed.