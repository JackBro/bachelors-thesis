\subsubsection{Mandtory Integrity Control}
Mandatory Integrity Control is based on the Bella-Padula model and multi level security. On the windows operating system, there are four different integrity levels with increasing privileges: low, medium, high, system. Figure \#\#\#\# shows an example of the integrity hierarchy. By definition it is not possible to access an object higher integrity level than the accessing object. Therefor, if process memory tampering is used, both processes need to have at least the same integrity level. The fourth integrity level system cannot be reached from the logged in user. Processes can now be started with one of the remaining integrity levels. By default, all processes receive medium integrity level. Low needs to be explicitly assigned and high is used only if the process is started with elevated permissions. Thus, as most processes run with medium integrity level, mandatory integrity control serves to be useless in preventing memory tampering. Even if chrome will receive high integrity level, this cannot be recommended as chrome is not trustworthy. If an attack is successful, the attacker will now have more access right than with the default medium integrity level.