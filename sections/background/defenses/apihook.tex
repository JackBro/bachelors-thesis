\subsubsection{Blocking DLL Injections by hooking \syscall{LoadLibrary}}
For the DLL injections based attacks of node [1.2] of Figure \ref{fig:attacks_external}, access control lists as explained in Sections \ref{sec:sd} and \ref{sec:dacl} can not be used, as it is unknown which DLLs are malicious. For which DLL files an ACL entry should be set to prevent execution. Additionally, the user of a file has always full permissions to change the ACL entry, so that any permission restriction can be easily undone. Thus, the ACL turns out to be useless and doesn't prevent loading of DLLs in all cases. 

Therefore a better approach is, hooking the \syscall{LoadLibrary} function, by detouring the function call, as shown in Figure \ref{fig:detours}. The application will then get called every time a DLL is loaded and can return early with a \syscall{STATUS\_ACCCESS\_DENIED} error code, if the to be loaded DLL is unwanted. However, this function might not be called in all DLL load cases, as the hook is placed after the process has started running. As a local hook is not enough to detect DLL loads before the hook is installed, a deeper system hook is needed. Unfortunately, this requires patching the system call table and is prevented on 64 bit systems due to PatchGuard since Windows 7. Therefore, a different approach needs to be taken to catch all cases.