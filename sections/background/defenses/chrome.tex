\subsubsection{\emph{Chrome} defenses}
\paragraph{\emph{Chrome ELF}}
\emph{Google Chrome} contains a \syscall{chrome\_elf.dll} file that exports important and security relevant functions. The main two purposes are caching of function addresses and a \gls{DLL} blacklist. \emph{Chrome ELF} is the first loaded module in \emph{Chrome} and can be used to add more security features like setting a process security descriptor\footnote{This concept is evaluated later in Chapter~\ref{sec:dacl}}. However, as \emph{Google's} development team states, they are not interested in protecting \emph{Chrome} from user-mode attacks \cite{chromium_security}. This is due to the reason, that a malicious process running under privilege level as Chrome can gain access and modify \emph{Chrome's} memory and settings. Instead, the user should rely on his installed anti-virus software to detect and prevent attacks.

\paragraph{Cached function addresses}
The second part of \emph{Chrome ELF's} features is caching of function addresses of \syscall{ntdll.dll}. All exported functions are stored inside a function lookup table to increase speed and reduce lookup operations of these often used functions. Additionally, the virtual memory space of the function lookup table is set to \syscall{PAGE\_READONLY}, to ensure that this memory can no longer be modified. This adds no security, because an attacker (external or internal) will remove the set \syscall{PAGE\_READONLY} flag and modify the lookup table. Additionally, this feature creates a new threat that can be used for attacks. Entries of the lookup table are changed to new addresses, making the described \gls{API} hooking attacks of Section~\ref{sec:attacks} even easier.

\paragraph{DLL Blacklist}
\emph{Chrome ELF} contains a \gls{DLL} blacklist feature, which prevents unwanted \glspl{DLL} from being loaded into \emph{Chrome}. The blacklist is split into two parts, the first list contained inside the code and the second list loaded from the local computers \emph{Registry}. The existing detection algorithms checks for an existing entry in the blacklist by comparing the file- and imagename of the loaded \gls{DLL} file. If a blacklisted \gls{DLL} is found, it is unloaded from \emph{Chrome}. A comparison between file- or imagenames is weak, because an attacker can change the name of the \gls{DLL} file. Additionally, every new build of the blacklisted \gls{DLL} will not be detected if the name is changed. The whole purpose of the existing blacklist feature is to increase stability of \emph{Google Chrome} and prevent unstable extensions and addons from running, to ensure an as optimal as possible user experience.