\subsubsection{Buffer Overflows}
Buffer overflows are among the most important security problems of modern applications \cite{pethia}, as they are easy to create \cite{bufferoverflows_easy}. Buffer overflows allow the attacker to execute arbitrary code inside the attacked application and as such are running under the same privilege level. Regardless the fact if code execution is possible, sensitive information might get revealed. One of the most severe examples of the past was the so called \emph{Heartbleed bug} \cite{durumeri}, which affected millions of servers worldwide, as it was present in the much used \emph{OpenSSL}\footnote{\url{https://www.openssl.org/}} library. Even though the attacker could not execute code, he was able to get sensitive information about the SSL certificate's private key. Encrypted data was no longer secure of external modifications or \emph{Man-in-the-Middle} attacks. Buffer overflow attacks represent the node [1.3] of Figure~\ref{fig:attacks_external} and are also a external modification. In contrast to the other previously shown attacks there are several countermeasures existing to mitigate the resulting exploit, which will be discussed in Chapter~\ref{sec:defenses}.