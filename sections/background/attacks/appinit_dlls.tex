\subsubsection{AppInit\_DLLs injection}
AppInit\_DLLs injection is the weakest kind of attack and can easily be mitigated. Windows provides a special registry key\footnote{The key can be found at: HKEY\_LOCAL\_MACHINE\textbackslash Software\textbackslash Microsoft\textbackslash Windows NT\textbackslash CurrentVersion\textbackslash Windows} that can be used to add one or multiple paths to DLL files, that should be automatically loaded into starting processes. The MSDN\cite{msdn_appinitdlls} lists the usage of User32.dll in the process as a requirement or otherwise the DLLs listed in the registry will not be loaded. Leaving out User32.dll is nearly impossible, however, since it is used in almost any process and one can assume that the given MSDN requirement is almost always fulfilled. Additionally a second key LoadAppInit\_DLLs has to be changed to 1 in order to activate AppInit\_DLLs injection. By default this value is set to 0 and cannot be changed without admin privileges. Defending against this kind of attack is comparably easy, by checking AppInit\_DLLs value and enumerating all loaded modules. If a match is found and unwanted it can be unloaded or as a safety measurement the application is terminated. As for Google Chrome, there is no validation currently in place and AppInit\_DLLs can be used to inject arbitrary DLL files.