\paragraph{\syscall{CreateRemoteThread} \gls{DLL} Injection}
\glspl{DLL} can be loaded into a target process by loading their path into the target memory with \gls{WPM} and creating a new thread via \syscall{CreateRemoteThread} that calls \syscall{LoadLibrary}. This kind of attack can be used on any process running under the same integrity level and therefore is widely used in game cheating.

\paragraph{Step 1} 
At first, a handle to the target process is requested via \syscall{OpenProcess} with \syscall{PROCESS\_VM\_WRITE} and \syscall{PROCESS\_VM\_OPERATION} access rights. These access rights are required to execute the \gls{WPM} function. If the permissions are missing, \gls{WPM} is not able to modify the virtual memory of the target process. Virtual memory protection flags do not have to be changed manually, as this is already happening inside \gls{WPM}. 

\paragraph{Step 2} 
A large enough amount of memory is allocated inside the target process with \syscall{VirtualAllocEx}, to hold the full path of the \gls{DLL}. 

\paragraph{Step 3}
Next, \gls{WPM} is used to transfer the \gls{DLL} path into the target memory space.

\paragraph{Step 4}
The injection can be completed by calling \syscall{CreateRemoteThread}, which loads the \gls{DLL} with the \syscall{LoadLibrary} function. The allocated memory segment is used as a parameter for the \syscall{LoadLibrary} function call. 

\paragraph{Step 5}
With the loaded \gls{DLL}, arbitrary code can get executed by the \gls{DLL} inside the attacked process, either via using \syscall{CreateRemoteThread} again or via the \gls{DLL}'s entry point.


An example of a basic \gls{DLL} injection using \gls{WPM} and \syscall{CreateRemoteThread} can be found in Appendix~\ref{appendix:writeprocessmemory}. \emph{Chrome} shows no existing defense mechanisms against direct memory modification.