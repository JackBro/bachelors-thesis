\subsubsection{Security Descriptor}
\label{sec:sd}
To create an access hierarchy, different permissions are created during the installation of a \emph{Windows} system. These include the current user account, which typically is running without elevated permissions, a group for administrator permissions and one user account containing the root permissions, also known as \syscall{NT-Authority} \cite{msdn_localsystem1, msdn_localsystem2}. \emph{Windows} allows to assign a special structure, the security descriptor, to various different objects. As different users have different access rights, multiple of these security descriptors are organized in an \gls{ACL}, which are by default denying access if it was not granted. Objects a security descriptor can be assigned to are \cite{msdn_sd}:
\begin{itemize}
\item Files or directories
\item Registry keys (Section~\ref{sec:registrykeys})
\item Named and anonymous pipes
\item Processes and threads
\item Access tokens
\item Windows Services
\item Job objects
\end{itemize}