\subsubsection{Dynamic-link library}
Dynamic-link libraries or in short DLLs are a very fundamental concept of the windows operating system and used in every running application. Programs can make use of DLLs by loading them into their virtual memory space and gaining access to the functionality exported by the DLLs. Every DLL contains an entry point function, which is called whenever the DLL is loaded into or unloaded from virtual memory. This function can be used for typical initialization or shutdown behavior. Additionally, most DLLs export functions that can be called from outside. This has major advantage. The exported behavior has to be shipped once, and can be used by many applications, which reduces the overall size of the program and removes unwanted redundancy.