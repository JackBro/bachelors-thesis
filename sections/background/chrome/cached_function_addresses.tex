\subsubsection{Cached function addresses}
The second part of Chrome ELFs features is caching of function addresses of \syscall{ntdll.dll}. All exported functions are stored inside a function lookup table to increase speed and reduce lookup operations of these often used functions. Additionally, the virtual memory space of the function lookup table is set to \syscall{PAGE\_READONLY}, to ensure that this memory can no longer be modified. This adds no security, because an attacker (external or internal) will remove the set \syscall{PAGE\_READONLY} flag and modify the lookup table. Additionally, this feature creates a new threat that can be used for attacks. Entries of the lookup table are changed to new addresses, making the described API hooking attacks of Section \ref{sec:attacks} even easier.