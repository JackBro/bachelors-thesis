\subsubsection{Chrome ELF DLL}
Google Chrome contains a \syscall{chrome\_elf.dll} file that exports  important and security relevant functions. The main two purposes are caching of function addresses and a \gls{DLL} blacklist. Chrome ELF is the first loaded module in Chrome and can be used to add more security features like setting a process security descriptor\footnote{This concept is evaluated later in chapter \ref{sec:dacl}}. However, as Google's development team states, they are not interested in protecting Chrome from user-mode attacks \cite{chromium_security}. This is due to the reason, that a malicious process running under privilege level as Chrome can gain access and modify Chrome's memory and settings. Instead, the user should rely on his installed anti-virus software to detect and prevent attacks.