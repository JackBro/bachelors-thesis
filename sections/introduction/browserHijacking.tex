\subsection{Problem}
"A browser hijacker is software that modifies the default browser behaviour" \cite{automatedspywarecollection}. That is a  vague definition but reflects one possible candidate of \gls{PMT} attacks. As a browser is required to view websites, it is used by over three billion \cite{cia} different users, making it a valuable attack target. An attack is able to change default settings like the search engine, which then gain the attacker additional ad revenues from more visitors. Traffic redirection proves to be very simple with \gls{PMT} and thus allows to make phishing activities look harmless, as they can no longer be distinguished from the original website. \gls{PMT} is hereby not limited to browsers and can be applied to any kind of software, e.g. games, browsers and other applications running on a computer. To apply a \gls{PMT} attack, the attacker does not need to be very skillful as required functions to access a process' virtual memory are shipped with the \emph{Windows} \gls{API}. The exploiting code is usually limited to less than 100 lines of code and allows arbitrary attacks on a process' memory. Defenses against \gls{PMT} are usually non existent, as anti virus software is not tracking such memory accesses, and the operating system is not providing further protection to applications.