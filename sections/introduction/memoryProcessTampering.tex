\subsection{Motivation}
\gls{PMT} describes the process of unwanted memory modifications of another process. These modifications can lead to unwanted behavior of the target application, resulting in several different security vulnerabilities in integrity, availability and confidentiality. The first, integrity, might no longer exist, as the intended behavior of the application is changed. Often applications are modified in such a way, that it is not possible to guarantee unchanged behavior. The second, availability, can no longer be ensured if the changes lead to longer run times. This results in a higher memory and CPU usage and the machine's limit is reached sooner than expected, leading to unexpected crashes or in general, service outages. The last, confidentiality, is broken, because the attacker can access the process' memory, thus allowing the attacker to read sensitive information like passwords or shared keys, and execution of injected code, altering the programs behavior. \gls{PMT} is present on every operating system and can be applied to any application running, with typical examples being \emph{games}. \gls{PMT} can be used to modify the games behavior, which is referred to as cheating, because the user gets an unfair advantage against other users.