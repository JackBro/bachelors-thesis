\section{Design and Implementation}
\label{chap_implementation}
The previous chapter showed that many different attacks still exists, that can't be prevented by the existing defenses. These attacks can be classified into two groups that have to be handled separately in order to increase security on the target platform. Attacks from outside the process virtual memory have to use WriteProcessMemory, which forms the first group and attacks from inside are relying on a DLL load during their initialization and execute their malicious code from inside, which form the second group. This classification makes it possible to design effective defenses that can't be circumvented. 

\subsection{Blocking WriteProcessMemory with a DACL}
\label{sec:implementation_dacl}
Most injection techniques call at some point in time the two important OpenProcess and subsequently WriteProcessMemory functions to inject code into or modify memory in the remote process. It is not possible to easily intercept these calls from inside a user-mode process, as the the process is suspended during the WriteProcessMemory function call. Instead, one might try to prevent this function call from happening and thus preventing the exploit by just limiting the process access rights and removing the required permissions to call virtual memory modifying functions like WriteProcessMemory and ReadProcessMemory. The OpenProcess function will then not be able to return a handle with the required permissions and calls to WriteProcessMemory will fail with STATUS\_ACCESS\_DENIED error code. Implementing this approach works great in a test-application seen in [INSERT REF HERE], however once added to Google Chrome, Chrome malfunctions and is no longer usable. The reason for that is Chrome's process hierarchy and the dependence on inter process communication with WriteProcessMemory and ReadProcessMemory. Compiling chromium with the shown DACL Deny code results in a running executable that is at the end not fully functioning. The content of the browser is not displayed and opening web pages fail, returning to the current open about:blank page immediately. Therefore a different approach has to be taken to limit virtual memory modification with WriteProcessMemory.

\subsection{Blocking WriteProcessMemory with a kernel mode driver}
The newly created problem with the modified DACL can be solved by creating a driver running in kernel-mode. As reading memory, checking files and running processes are a common problem when writing anti virus software, Microsoft introduced object manager routines since Windows Vista to provide an efficient way of reacting onto these events. A driver will then use the ObRegisterCallbacks function to get notified whenever processes are created, files are opened and images (DLLs and EXEs) are loaded. This functionality offers the programmer to create a pre- and post-execution callback function, that gets called whenever a process tries to receive a process handle. Microsoft offers an example driver using ObRegisterCallbacks at \cite{github_obcallback} and shows how to modify the requested process handle permissions to prevent termination through other applications such as the task manager. In order to protect chrome, the same actions described earlier will now be taken from inside the driver. Newly created process handles will be restricted to non virtual memory modifying permissions and the existing problem with inter process communication can be resolved by not restricting chromes child processes explicitly and giving them a full access handle. \ref{sec:appendix} shows an implemented driver to fulfill the requirements. A demonstration video of the resulting implementation can be found in the contained digital materials.

However, a naive implementation that just checks for process' executable file name may result in an access gain for the attacker. The injecting process simply has to be named chrome.exe and can gain access to the original chrome.exe. However the implementation restricts outside access regardless of the given process name successfully. Children and parent processes ids (pids) are stored inside two separate tables that are on access checked if they contain target and source process ids. The driver is using a dynamic list structure, so that an unlimited number of processes can be tracked. Finally the check for access rights can be done by finding a given pid inside the defined lists and returning a hash value which uniquely identifies the process. A good hash for this tree is the parent pid, as the process hierarchy is tracked inside the list. Only if both find operations result in a match of the returned pids, access is granted and no restriction is applied. Entries of the list (parent or child process) will get removed as soon as they exit, so that the remaining list is kept clean from garbage and reassignments of the same pid don't lead to a security hole. 

\subsection{Blocking DLL Injection by hooking LoadLibrary}
For the second classification group, the DLL based injections, DACL can't be used, as it is unknown which DLLs are malicious and thus, for which DLL files a DACL should to prevent execution. Additionally, the user of a file always has full permissions to change the DACL entry, so that any permission restriction can be easily undone. Thus, the DACL turns out to be useless and doesn't prevent loading of DLLs in all cases. Therefore a more interesting approach is hooking the LoadLibrary call, by detouring the function call, as shown in figure \ref{fig_detours}. The application will then get called every time a DLL is loaded and can return early with a STATUS\_ACCCESS\_DENIED error code, if the to be loaded DLL is unwanted. However, this function might not be called in all DLL load cases and only the underlying LdrLoadDll is a reliable function to hook. Unfortunately, this requires patching the system call table and is prevented on x64 bit systems due to PatchGuard since Windows 7. Therefore, a different approach needs to be taken to catch all cases.

\subsection{Blocking DLL Injection with a kernel mode driver}
As previously stated, the resulting implementation already uses a kernel mode driver to prevent WriteProcessMemory calls. It is therefore a suitable start point to prevent DLL injections without being required to hook into the kernel. The Windows API for drivers offers the existing PsSetLoadImageNotfiyRoutine function call, to register a callback function that is whenever a DLL is mapped into memory. During the function call, the process is in a suspended state and thus execution of the DLL hasn't begun. However, this entry point for DLL load prevention is not optimal, as this function call occurs after the DLL was already mapped into memory and a pre loading callback would be better to use in this situations. Unfortunately, this callback function does not exist so far, so the implementation has to rely on the post load callback. To make use of this function, the MSDN documentation has to be read carefully, as not following it will lead to a deadlock or even make the system crash in a blue screen of death. Most notably is the fact that the function call occurs on IRQ Level PASSIVE\_LEVEL and special kernel mode APCs are disabled. This is at first a major limitation naive implementations. Opening and reading file calls will not complete, as the underlying APC event is not sent, thus in a naive implementation, reading a file seems impossible at a first glance. The reason for the disabled special kernel mode APCs is a previous call to the KeEnterGuardedRegion function, so a call to KeLeaveGuardedRegion can enable APC events again. However, this shouldn't be done here, as the result turns out to be unpredictable and leading to random deadlocks or failing accesses on files. By digging further a solution can be obtained by using a second running thread, a so called work item, which execution is handled separately by the system. The advantage of that is the enabled special kernel mode APCs and the possibility to freely change the IRQL during execution. One major disadvantage of this way is that some sort of thread synchronization between the work item and the initial callback function has to be introduced and thus resulting into driver typical busy waiting, which is still faster than non busy waiting and eventually occurring context changes. A file handle can now be obtained and from the file data a sha256 hash is generated, which is used later to compare it to a known whitelist of dll files. Although gaining a file handle seems easy, it is in fact not! The Windows API offers several different ways of obtaining a valid file handle with functions like ZwOpenFile or ObOpenObjectByPointer. The second function ObOpenObjectByPointer will generate a handle from a given pointer to an object. Luckily, that's exactly what can be found inside the input parameters of the PsSetLoadImageNotifyRoutine callback. The image info can be extended to obtain a file object, and the ObOpenObjectByPointer function can then be used to obtain a handle from this file object. Though this didn't work in all cases and lead to "random" results. If a file was opened a first time, ObOpenObjectByPointer would succeed and return a valid file handle. Subsequent calls to the same file lead to error STATUS\_UNSUCCESSFUL for no obvious reason. Kernel-debugging allowed to find the reason of failure, which was an internal call to ObpIncrementHandleCount that had returned STATUS\_UNSUCCESFUL. There can be found nothign with respect to this problem on the internet and might be a bug inside the windows kernel itself. Therefore, ObOpenObjectByPointer can't be used and only ZwOpenFile works as expected, but gives a new problem, the full file path. That's surprising, as typical callback routines always get the full filepath as input parameter, PsSetLoadImageNotifyRoutine does not! Some path constructions is therefore required, but will overall work and allow reading of all loaded DLL files. Finally, the resulting sha256 hash can be calculated and compared to the whitelist and if no match is found, actions can be taken prevent the DLL from executing. An initial thought could be to undo the internal ZwMapViewOfSection/NtMapViewOfSection function by calling it's counterpair ZwUnmapViewOfSection/NtUnmapViewOfSection and thus removing the DLL from memory. The idea is great, but will not succeed, as a special lock is hold, the LdrLoadLoaderLock, that can't be accessed at this point of code. Calling the functions regardless will deadlock the driver and soon after deadlock the whole system. Thus, the DLL will only get patched, rendering it, although still residing in memory, useless. To do that, three points in memory receive modifications: the entrypoint address, the DOS MZ header and the entrypoint function. The entrypoint address is set to NULL, the entrypoint function will get patched with a single ret (0xC3) instruction and the memory containg the DOS MZ header will get zeroed. All three steps guarantee, that nothing will be able to execute from this memory mapped DLL file and thus preventing the typical DLL injections from SetWindowsHookEx, the registry or other techniques.